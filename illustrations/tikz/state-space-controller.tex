\documentclass[tikz,dvisvgm]{standalone}
\usetikzlibrary{decorations.markings}
\usetikzlibrary{decorations.text}
\usetikzlibrary{shapes,arrows}

\tikzstyle{block} = [draw, fill=white, rectangle, minimum height=3em, minimum width=6em]
\tikzstyle{gain} = [draw, fill=white, isosceles triangle, node distance=2cm]
\tikzstyle{sum} = [draw, fill=white, circle, node distance=2cm]

\begin{document}
	\begin{tikzpicture}[node distance=1.5cm, >=latex']
		\node[coordinate] (input) at (2, 0) {};
        \node[gain, right of=input, node distance=1cm] (ctlgain) {$Gr$};
		\node[sum, right of=ctlgain] (sub) {};
		\node[block, right of=sub, node distance=2cm] (stateequ) {$\dot x = Ax + Bu$};
        \node[coordinate, right of=stateequ] (feedback) {};
        \node[block, right of=feedback] (outputequ) {$y = Cx$};
        \node[coordinate, right of=outputequ] (output) {};
        \node[gain, below of=stateequ, shape border rotate=180] (ctlfeedback){$Kx$};

        \draw[->] (input) -- node [above] {$r$} (ctlgain);
        \draw[->] (ctlgain) -- (sub);
        \draw[->] (sub) -- node [above] {$u$} (stateequ);
        \draw[-] (stateequ) -- (feedback);
        \draw[->] (feedback) -- node[above] {$x$} (outputequ);
        \draw[->] (outputequ) -- node[above] {$y$} (output);
        \draw[->] (feedback) |- (ctlfeedback);
        \draw[->] (ctlfeedback) -| node[pos=0.99, below left] {$-$} (sub);

        \node[coordinate] (input2) at (0, -5) {};
        \node[sum, right of=input2, node distance=1cm] (error2) {};
        \node[block, right of=error2, minimum width=3em] (preint2) {$\int$};
        \node[gain, right of=preint2] (errorgain2) {$F_2 z$};
        \node[sum, right of=errorgain2] (sub2) {};
        \node[block, right of=sub2, node distance=2cm] (stateequ2) {$\dot x = Ax + Bu$};
        \node[coordinate, right of=stateequ2] (feedback2) {};
        \node[block, right of=feedback2] (outputequ2) {$y = Cx$};
        \node[coordinate, right of=outputequ2] (output2) {};
        \node[coordinate, right of=output2, node distance=1cm] (output3) {};
        \node[gain, below of=stateequ2, shape border rotate=180] (stategain2) {$F_1 x$};
        \node[coordinate, below of=stategain2] (waypoint2) {};

        \draw[->] (input2) -- node [above] {$r$} (error2);
        \draw[->] (error2) -- (preint2);
        \draw[->] (preint2) -- node [above] {$z$}(errorgain2);
        \draw[->] (errorgain2) -- node [pos=0.99, above left] {$-$}(sub2);
        \draw[->] (sub2) -- node [above] {$u$} (stateequ2);
        \draw[-] (stateequ2) -- (feedback2);
        \draw[->] (feedback2) -- node [above] {$x$} (outputequ2);
        \draw[-] (outputequ2) -- (output2);
        \draw[->] (output2) -- node [above] {$y$} (output3);

        % State feedback
        \draw[->] (feedback2) |- (stategain2);
        \draw[->] (stategain2) -| node [pos=0.99, below left] {$-$}(sub2);

        % Error feedback
        \draw[->] (output2) |- (waypoint2) -| node [pos=0.99, below left] {$-$} (error2);


	\end{tikzpicture}
\end{document}